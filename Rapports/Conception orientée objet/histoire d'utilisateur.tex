\documentclass[11pt,a4paper]{article}
\usepackage[latin1]{inputenc}
\usepackage[french]{babel}
\usepackage[T1]{fontenc}
\usepackage{amsmath}
\usepackage{amsfonts}
\usepackage{amssymb}
\author{Groupe M}
\title{Histoires utilisateurs}
\begin{document}
\maketitle

\section*{Client}
\begin{itemize}
\item En tant que \textit{client} utilisant l'application pour la premi�re fois, lors du d�marrage de l'application, afin de pouvoir acceder aux fonctionalit�s de l'application je dois m'inscrire.
\end{itemize}

\section*{Utilisateur}
\begin{itemize}
\item En tant que \textit{utilisateur} utilisant l'application, lors du d�marrage de l'application, afin de pouvoir acceder aux fonctionalit�s de l'application je dois introduire un login ou un mot de passe.
\item En tant que \textit{utilisateur} utilisant l'application, afin de ajouter un element a ma commande, je peux consulter la carte et acceder au details de chaque produit afin de faire mon choix.
\item En tant que \textit{utilisateur} utilisant l'application, afin de consulter/modifier ma commande, je peux acceder a ma commande et la modifier.
\item En tant que \textit{utilisateur} utilisant l'application, afin de passer ma commande, je peux acceder a ma commande et cliquer sur "commander".
\item En tant que \textit{utilisateur} utilisant l'application, afin de consulter mon addition, je peux acceder a mon addition.
\item En tant que \textit{utilisateur} utilisant l'application, afin de preciser a quelle table je suis, je peux l'ajouter dans ma commande.
\item En tant que \textit{utilisateur} utilisant l'application, afin de modifier mon compte, je peux acceder mon compte.
\item En tant que \textit{utilisateur} utilisant l'application, afin de donner mon avis, je peux donner une note a un produit.
\end{itemize}

\section*{Serveur}
\begin{itemize}
\item En tant que \textit{serveur} utilisant l'application, afin de servir les commandes, je peux consulter la liste des commandes et savoir a quelle table cette commande est prevue.
\item En tant que \textit{serveur} utilisant l'application, afin de fournir une facture a un client, je peux avoir acc�s aux commandes de la table et voir le total.
\item En tant que \textit{serveur} utilisant l'application, afin de voir l'�tat du stock, je peux consulter l'�tat du stock.
\end{itemize}

\section*{Administrateur}
\begin{itemize}
\item En tant que \textit{administrateur} utilisant l'application, afin de voir l'�tat du stock, je peux consulter l'�tat du stock, le modifier, recevoir une alerte si un item passe sous le seuil que j'aurais definis et passer des commandes.
\item En tant que \textit{administrateur} utilisant l'application, afin d'ajouter un serveur, je peux creer un compte serveur.
\item  En tant que \textit{administrateur} utilisant l'application, afin de consulter les statistiques du bar, je peux consulter les statistiques.
\item En tant que \textit{administrateur} utilisant l'application,afin de gerer les produits, je peut modifier les valeurs des produits (nom, prix, description, etc...)
\end{itemize}
\end{document}